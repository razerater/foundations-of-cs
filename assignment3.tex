\documentclass[fleqn]{article}

\usepackage{amsmath} % for equations
\usepackage{amssymb} % for symbols
\usepackage[margin=0.75in]{geometry} % for setting margin
%\usepackage{tikz} % for drawing
%\usepackage{verbatim} % for multiline comments
\usepackage{color} % for colored proofs
\usepackage{parskip} % looks nice
\usepackage{scrextend} % for block indentation

\title{Assignment 3}
\author{Raz Reed}
\date{September 20, 2017}

\begin{document}
\pagenumbering{gobble}
\maketitle

\newpage
{\noindent\Large\bf Problem 5.4}\vspace{1em}\par
\textbf{(i)} Not valid. When proving $P(n) \rightarrow P(n+1)$, you may not assume that what you are trying to prove is true.\par
\textbf{(ii)} Valid. When proving $P(n) \rightarrow P(n+1)$, you may assume $P(n+1)$ is false in order to establish a proof by contradiction.

\newpage
{\noindent\Large\bf Problem 5.3}\vspace{1em}\par
\textbf{(c)} $P(2)$ is T and $P(n) \rightarrow (P(n^2) \land P(n-2))$ is T for $n \geq 2$.
\begin{equation*}
	P(2) \rightarrow P(4) \rightarrow P(16) \rightarrow P(256) \rightarrow\ ...
\end{equation*}
\begin{equation*}
	P(2) \leftarrow P(4) \leftarrow P(6) \leftarrow P(8) \leftarrow\ ...
\end{equation*}
\par$P(n)$ is T for $n \geq 2$.

\newpage
{\noindent\Large\bf Problem 5.11}\vspace{1em}\par
\textbf{(b)} $P(n): 1+\frac{1}{2}ln(n) \leq H_n \leq 1+ln(n)$.\par
Claim: $P(n)$ is T for all $n > 0$.\par
Proof:
\begin{addmargin}{2em}
	Base case: $P(1)$ is T.\par
	Induction: assume $P(n)$, prove $P(n+1): 1+\frac{1}{2}ln(n+1) \leq H_{n+1} \leq 1+ln(n+1)$.
	\begin{align*}
		&H_n = 1+\frac{1}{2}+\frac{1}{3}+...+\frac{1}{n}\\
		&H_{n+1} = 1+\frac{1}{2}+\frac{1}{3}+...+\frac{1}{n}+\frac{1}{n+1}\\
		&H_{n+1} = H_n+\frac{1}{n+1}
	\end{align*}
	We know that for $0 \leq x \leq \frac{1}{2}, -2x \leq ln(1-x) \leq -x$.\par
	If we plug in $\frac{1}{n+1}$ for $x$, we get:
	\begin{align*}
		&x = \frac{1}{n+1}\\
		&\frac{-2}{n+1} \leq ln(1-\frac{1}{n+1}) \leq \frac{-1}{n+1}\\
		&ln(1-\frac{1}{n+1}) = ln(\frac{n+1}{n+1}-\frac{1}{n+1}) = ln(\frac{n}{n+1}) = ln(n)-ln(n+1)\\
		&\frac{-2}{n+1} \leq ln(n)-ln(n+1) \leq \frac{-1}{n+1}\\
		&\frac{2}{n+1} \geq ln(n+1)-ln(n) \geq \frac{1}{n+1}\\
		&\frac{1}{n+1} \leq ln(n+1)-ln(n) \leq \frac{2}{n+1}
	\end{align*}
	Take the right part of that inequality:
	\begin{align*}
		&ln(n+1)-ln(n) \leq \frac{2}{n+1}\\
		&\frac{1}{2}ln(n+1)-\frac{1}{2}ln(n) \leq \frac{1}{n+1}\\
		&\frac{1}{2}ln(n+1) \leq \frac{1}{2}ln(n)+\frac{1}{n+1}\\
		&1+\frac{1}{2}ln(n+1) \leq 1+\frac{1}{2}ln(n)+\frac{1}{n+1} \leq H_n+\frac{1}{n+1} = H_{n+1}\\
		&{\color{green}1+\frac{1}{2}ln(n+1) = H_{n+1}}
	\end{align*}
	We just proved the first part of $P(n+1)$. Now take the left part of the inequality:
	\begin{align*}
		&\frac{1}{n+1} \leq ln(n+1)-ln(n)\\
		&ln(n+1)-ln(n) \geq \frac{1}{n+1}\\
		&ln(n+1) \geq \frac{1}{n+1}+ln(n)\\
		&1+ln(n+1) \geq ln(n)+1+\frac{1}{n+1} \geq H_n+\frac{1}{n+1} = H_{n+1}\\
		&{\color{green}H_{n+1} \leq 1+ln(n+1)}\\
		&\therefore P(n+1) \text{ is T.}
	\end{align*}
	Therefore, $P(n)$ is true for all $n > 0$.
\end{addmargin}

\newpage
{\noindent\Large\bf Problem 5.43}\vspace{1em}\par
\textbf{(a)} Allow $(x,y)$ to represent the x- and y-coordinate of the robot. Also treat the starting position of the robot as the origin $(0,0)$ of the grid. There are four directions the robot can move, and each move modifies its position as dictated by this table:\vspace{0.35cm}\par
\begin{tabular}{c|l|c|c|c}
	old position & \multicolumn{1}{c|}{move} & new position & change in $(x,y)$ & change in $x$ + change in $y$
	\tabularnewline\hline
	$(x,y)$ & down-left & $(x-1,y-1)$ & $(-1,-1)$ & -2\\
	$(x,y)$ & down-right & $(x+1,y-1)$ & $(1,-1)$ & 0\\
	$(x,y)$ & up-left & $(x-1,y+1)$ & $(-1,1)$ & 0\\
	$(x,y)$ & up-right & $(x+1,y+1)$ & $(1,1)$ & 2
\end{tabular}
\vspace{0.35cm}\par
The sum of the x- and y-coordinate of the robot $x+y$ is 0 at the start. Adding any even number to an even number will result in an even number. The robot's $x+y$ is even, and any move it makes will add an even number (-2, 0, or 2) to its $x+y$. Therefore, its $x+y$ will always be even.\par
The shaded square is at $(1,0)$. Its $x+y$ is 1, an odd number. The robot's $x+y$ will always be even, therefore no sequence of moves takes the robot to the shaded square.\vspace{1em}\par

\textbf{(b)} The robot now cannot move up one square and right one square. It has a new ability to move up two squares and right one square (in one move). It has all the other moves it had previously. I will demonstrate that this effectively grants it the ability to move in any orthogonal direction.\par
For simplicity, assume that:
\begin{itemize}
	\item DL = robot moves down one and left one
	\item DR = robot moves down one and right one
	\item UL = robot moves up one and left one
	\item UUR = robot moves up \textbf{two} and right one
	\item U = robot moves up one
	\item D = robot moves down one
	\item L = robot moves left one
	\item R = robot moves right one
	\item UR = robot moves up one and right one
\end{itemize}
The last five moves can't be done by the robot in one step, but they can be achieved through series of steps listed in the third column of this table:\vspace{0.35cm}\par
% note to self: in the tabular environment,
% \newline ends the line inside the cell
% \tabularnewline ends the row of the table
\begin{tabular}{c|c|p{4cm}|c}
	old position & new move & series of moves used to achieve the new move & change in $(x,y)$
	\tabularnewline\hline
	$(x,y)$ & U & DL, UUR & $(x,y+1)$\\
	$(x,y)$ & D & DL, DL, DR, UUR & $(x,y-1)$\\
	$(x,y)$ & L & DL, DL, UUR & $(x-1,y)$\\
	$(x,y)$ & R & DL, DR, UUR & $(x+1,y)$\\
	$(x,y)$ & UR & DL, UUR, DL, DR, UUR & $(x+1,y+1)$\\
\end{tabular}
\vspace{0.35cm}\par
The robot can travel from one square to any orthogonally or diagonally adjacent square by a finite sequence of moves. Therefore, it can reach any square $(m,n)$ on the infinite grid by a finite sequence of moves.

\newpage
{\noindent\Large\bf Exercise 6.2}\vspace{1em}\par
Claim: $P(n): n^3 < 2^n\quad \forall n \geq 10$.\par
Proof:
\begin{addmargin}{2em}
	Base case: $P(10)$ is T.\par
	Induction: assume $P(n)$, prove $P(n+1): (n+1)^3 < 2^{n+1}\quad \forall n \geq 10$.
	\begin{align*}
		&n^3 < 2^n\\
		&(n+1)^3 = n^3+3n^2+3n+1\\
		&n^3+3n^2+3n+1 < 2^n+3n^2+3n+1\\
		&(n+1)^3 < 2^n+3n^2+3n+1
	\end{align*}
	Stuck; we must create a new hypothesis: $Q(n): n^3 < 2^n \land 3n^2+3n+1 < 2^n\quad \forall n \geq 10$.\par
	Base case: $Q(10)$ is T.\par
	Induction: assume $Q(n)$, prove $Q(n+1): (n+1)^3 < 2^{n+1} \land 3n^2+9n+7 < 2^{n+1}\quad \forall n \geq 10$.
	\begin{align*}
		&n^3 < 2^n\\
		&3n^2+3n+1 < 2^n\\
		&n^3+3n^2+3n+1 < n^3+2^n < 2^n+2^n = 2^{n+1}\\
		&{\color{green}(n+1)^3 < 2^{n+1}}
	\end{align*}
	We proved the first half of $Q(n)$ but we're stuck on the second; we must create a new hypothesis: $R(n): n^3 < 2^n \land 3n^2+3n+1 < 2^n \land 6n+6 < 2^n\quad \forall n \geq 10$.\par
	Base case: $R(10)$ is T.\par
	Induction: assume $R(n)$, prove $R(n+1): (n+1)^3 < 2^{n+1} \land 3n^2+9n+7 < 2^{n+1} \land 6(n+1)+6 < 2^{n+1}\quad \forall n \geq 10$.
	\begin{align*}
		&{\color{green}(n+1)^3 < 2^{n+1}} \text{ (proven already)}\\
		&6n+6 < 2^n\\
		&3n^2+3n+1 < 2^n\\
		&3n^2+3n+1+6n+6 < 2^n+6n+6 < 2^n+2^n = 2^{n+1}\\
		&{\color{green}3n^2+9n+7 < 2^{n+1}}\\\\
		&6n+6 < 2^n\\
		&6 < 6n+6\\
		&6 < 2^n\\
		&6n+6+6 < 2^n+6 < 2^n+2^n = 2^{n+1}\\
		&{\color{green}6n+12 < 2^{n+1}}\\
		&R(n+1) \text{ is T.}\\
		&R(n) \rightarrow R(n+1)\\
		&\therefore Q(n) \rightarrow Q(n+1)\\
		&\therefore P(n) \rightarrow P(n+1)
	\end{align*}
	Therefore, the claim $P(n)$ is true for all $n \geq 10$.
\end{addmargin}

\newpage
{\noindent\Large\bf Exercise 6.4}\vspace{1em}\par
Claim: if the missing square is at position $(n,n)$ in the $2^n \times 2^n$ grid, the patio can still be $L$-tiled.\par
Stronger claim: $P(n):$ if there is a missing square on the $2^n \times 2^n$ patio at any location, the patio can still be $L$-tiled.\par
Proof:
\begin{addmargin}{2em}
	Base case: a $2 \times 2$ grid can be $L$-tiled for every possible location of the black square.\par
	Induction: assume $P(n)$, prove $P(n+1)$.\par
	Take a $2^{n+1} \times 2^{n+1}$ grid and split it into four sections. $L$-tile the center such that you do not tile the section containing the missing square. Each of these sections is now a $2^n \times 2^n$ grid with a missing square. Because $P(n)$ is T, we know that we can $L$-tile each of these sections and thereby the whole grid. $P(n+1)$ is T, so $P(n) \rightarrow P(n+1)$, therefore $P(n)$ is true for all $n \geq 1$.
\end{addmargin}
\end{document}